% Created 2015-01-21 Wed 13:38
\documentclass[11pt]{article}
\usepackage[utf8]{inputenc}
\usepackage[T1]{fontenc}
\usepackage{fixltx2e}
\usepackage{graphicx}
\usepackage{longtable}
\usepackage{float}
\usepackage{wrapfig}
\usepackage{rotating}
\usepackage[normalem]{ulem}
\usepackage{amsmath}
\usepackage{textcomp}
\usepackage{marvosym}
\usepackage{wasysym}
\usepackage{amssymb}
\usepackage{hyperref}
\tolerance=1000
\date{10-01-2015}
\title{small rnaseq-methods}
\begin{document}

\maketitle

\section{Methods}

All samples are processed using sRNA-seq pipeline implemented in
\href{http://bcbio-nextgen.readthedocs.org/en/latest/}{bcbio-nextgen project}.
Raw reads will be examined for quality
issues using FastQC to ensure library generation and sequencing are
suitable for further analysis. 3' end adapter were trimmed from reads using
cutadapt \cite{Martin:2011va}. Trimmed reads were aligned to miRBase
21 \cite{Kozomara2014} using seqbuster \cite{Pantano2010}.

miRNAs counting was done with isomiRs package \href{http://github.com/lpantano/isomiRs} discarding any sequence with only 1 count. A second round of filtering was applied to remove several of the low expressors; a minimum of 2 replicates per condition in every case were required to have 3 or more counts. Normalilzation and differential expression at the miRNA level were called with
DESeq2 \cite{Love:2014do}, which has been shown to be a robust,
conservative differential expression caller.

\section{Bibliography}
\bibliographystyle{plain}
\bibliography{srnaseq-methods}
\end{document}
